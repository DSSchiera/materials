\documentclass[11pt]{article}

    \usepackage[breakable]{tcolorbox}
    \usepackage{parskip} % Stop auto-indenting (to mimic markdown behaviour)
    
    \usepackage{iftex}
    \ifPDFTeX
    	\usepackage[T1]{fontenc}
    	\usepackage{mathpazo}
    \else
    	\usepackage{fontspec}
    \fi

    % Basic figure setup, for now with no caption control since it's done
    % automatically by Pandoc (which extracts ![](path) syntax from Markdown).
    \usepackage{graphicx}
    % Maintain compatibility with old templates. Remove in nbconvert 6.0
    \let\Oldincludegraphics\includegraphics
    % Ensure that by default, figures have no caption (until we provide a
    % proper Figure object with a Caption API and a way to capture that
    % in the conversion process - todo).
    \usepackage{caption}
    \DeclareCaptionFormat{nocaption}{}
    \captionsetup{format=nocaption,aboveskip=0pt,belowskip=0pt}

    \usepackage{float}
    \floatplacement{figure}{H} % forces figures to be placed at the correct location
    \usepackage{xcolor} % Allow colors to be defined
    \usepackage{enumerate} % Needed for markdown enumerations to work
    \usepackage{geometry} % Used to adjust the document margins
    \usepackage{amsmath} % Equations
    \usepackage{amssymb} % Equations
    \usepackage{textcomp} % defines textquotesingle
    % Hack from http://tex.stackexchange.com/a/47451/13684:
    \AtBeginDocument{%
        \def\PYZsq{\textquotesingle}% Upright quotes in Pygmentized code
    }
    \usepackage{upquote} % Upright quotes for verbatim code
    \usepackage{eurosym} % defines \euro
    \usepackage[mathletters]{ucs} % Extended unicode (utf-8) support
    \usepackage{fancyvrb} % verbatim replacement that allows latex
    \usepackage{grffile} % extends the file name processing of package graphics 
                         % to support a larger range
    \makeatletter % fix for old versions of grffile with XeLaTeX
    \@ifpackagelater{grffile}{2019/11/01}
    {
      % Do nothing on new versions
    }
    {
      \def\Gread@@xetex#1{%
        \IfFileExists{"\Gin@base".bb}%
        {\Gread@eps{\Gin@base.bb}}%
        {\Gread@@xetex@aux#1}%
      }
    }
    \makeatother
    \usepackage[Export]{adjustbox} % Used to constrain images to a maximum size
    \adjustboxset{max size={0.9\linewidth}{0.9\paperheight}}

    % The hyperref package gives us a pdf with properly built
    % internal navigation ('pdf bookmarks' for the table of contents,
    % internal cross-reference links, web links for URLs, etc.)
    \usepackage{hyperref}
    % The default LaTeX title has an obnoxious amount of whitespace. By default,
    % titling removes some of it. It also provides customization options.
    \usepackage{titling}
    \usepackage{longtable} % longtable support required by pandoc >1.10
    \usepackage{booktabs}  % table support for pandoc > 1.12.2
    \usepackage[inline]{enumitem} % IRkernel/repr support (it uses the enumerate* environment)
    \usepackage[normalem]{ulem} % ulem is needed to support strikethroughs (\sout)
                                % normalem makes italics be italics, not underlines
    \usepackage{mathrsfs}
    

    
    % Colors for the hyperref package
    \definecolor{urlcolor}{rgb}{0,.145,.698}
    \definecolor{linkcolor}{rgb}{.71,0.21,0.01}
    \definecolor{citecolor}{rgb}{.12,.54,.11}

    % ANSI colors
    \definecolor{ansi-black}{HTML}{3E424D}
    \definecolor{ansi-black-intense}{HTML}{282C36}
    \definecolor{ansi-red}{HTML}{E75C58}
    \definecolor{ansi-red-intense}{HTML}{B22B31}
    \definecolor{ansi-green}{HTML}{00A250}
    \definecolor{ansi-green-intense}{HTML}{007427}
    \definecolor{ansi-yellow}{HTML}{DDB62B}
    \definecolor{ansi-yellow-intense}{HTML}{B27D12}
    \definecolor{ansi-blue}{HTML}{208FFB}
    \definecolor{ansi-blue-intense}{HTML}{0065CA}
    \definecolor{ansi-magenta}{HTML}{D160C4}
    \definecolor{ansi-magenta-intense}{HTML}{A03196}
    \definecolor{ansi-cyan}{HTML}{60C6C8}
    \definecolor{ansi-cyan-intense}{HTML}{258F8F}
    \definecolor{ansi-white}{HTML}{C5C1B4}
    \definecolor{ansi-white-intense}{HTML}{A1A6B2}
    \definecolor{ansi-default-inverse-fg}{HTML}{FFFFFF}
    \definecolor{ansi-default-inverse-bg}{HTML}{000000}

    % common color for the border for error outputs.
    \definecolor{outerrorbackground}{HTML}{FFDFDF}

    % commands and environments needed by pandoc snippets
    % extracted from the output of `pandoc -s`
    \providecommand{\tightlist}{%
      \setlength{\itemsep}{0pt}\setlength{\parskip}{0pt}}
    \DefineVerbatimEnvironment{Highlighting}{Verbatim}{commandchars=\\\{\}}
    % Add ',fontsize=\small' for more characters per line
    \newenvironment{Shaded}{}{}
    \newcommand{\KeywordTok}[1]{\textcolor[rgb]{0.00,0.44,0.13}{\textbf{{#1}}}}
    \newcommand{\DataTypeTok}[1]{\textcolor[rgb]{0.56,0.13,0.00}{{#1}}}
    \newcommand{\DecValTok}[1]{\textcolor[rgb]{0.25,0.63,0.44}{{#1}}}
    \newcommand{\BaseNTok}[1]{\textcolor[rgb]{0.25,0.63,0.44}{{#1}}}
    \newcommand{\FloatTok}[1]{\textcolor[rgb]{0.25,0.63,0.44}{{#1}}}
    \newcommand{\CharTok}[1]{\textcolor[rgb]{0.25,0.44,0.63}{{#1}}}
    \newcommand{\StringTok}[1]{\textcolor[rgb]{0.25,0.44,0.63}{{#1}}}
    \newcommand{\CommentTok}[1]{\textcolor[rgb]{0.38,0.63,0.69}{\textit{{#1}}}}
    \newcommand{\OtherTok}[1]{\textcolor[rgb]{0.00,0.44,0.13}{{#1}}}
    \newcommand{\AlertTok}[1]{\textcolor[rgb]{1.00,0.00,0.00}{\textbf{{#1}}}}
    \newcommand{\FunctionTok}[1]{\textcolor[rgb]{0.02,0.16,0.49}{{#1}}}
    \newcommand{\RegionMarkerTok}[1]{{#1}}
    \newcommand{\ErrorTok}[1]{\textcolor[rgb]{1.00,0.00,0.00}{\textbf{{#1}}}}
    \newcommand{\NormalTok}[1]{{#1}}
    
    % Additional commands for more recent versions of Pandoc
    \newcommand{\ConstantTok}[1]{\textcolor[rgb]{0.53,0.00,0.00}{{#1}}}
    \newcommand{\SpecialCharTok}[1]{\textcolor[rgb]{0.25,0.44,0.63}{{#1}}}
    \newcommand{\VerbatimStringTok}[1]{\textcolor[rgb]{0.25,0.44,0.63}{{#1}}}
    \newcommand{\SpecialStringTok}[1]{\textcolor[rgb]{0.73,0.40,0.53}{{#1}}}
    \newcommand{\ImportTok}[1]{{#1}}
    \newcommand{\DocumentationTok}[1]{\textcolor[rgb]{0.73,0.13,0.13}{\textit{{#1}}}}
    \newcommand{\AnnotationTok}[1]{\textcolor[rgb]{0.38,0.63,0.69}{\textbf{\textit{{#1}}}}}
    \newcommand{\CommentVarTok}[1]{\textcolor[rgb]{0.38,0.63,0.69}{\textbf{\textit{{#1}}}}}
    \newcommand{\VariableTok}[1]{\textcolor[rgb]{0.10,0.09,0.49}{{#1}}}
    \newcommand{\ControlFlowTok}[1]{\textcolor[rgb]{0.00,0.44,0.13}{\textbf{{#1}}}}
    \newcommand{\OperatorTok}[1]{\textcolor[rgb]{0.40,0.40,0.40}{{#1}}}
    \newcommand{\BuiltInTok}[1]{{#1}}
    \newcommand{\ExtensionTok}[1]{{#1}}
    \newcommand{\PreprocessorTok}[1]{\textcolor[rgb]{0.74,0.48,0.00}{{#1}}}
    \newcommand{\AttributeTok}[1]{\textcolor[rgb]{0.49,0.56,0.16}{{#1}}}
    \newcommand{\InformationTok}[1]{\textcolor[rgb]{0.38,0.63,0.69}{\textbf{\textit{{#1}}}}}
    \newcommand{\WarningTok}[1]{\textcolor[rgb]{0.38,0.63,0.69}{\textbf{\textit{{#1}}}}}
    
    
    % Define a nice break command that doesn't care if a line doesn't already
    % exist.
    \def\br{\hspace*{\fill} \\* }
    % Math Jax compatibility definitions
    \def\gt{>}
    \def\lt{<}
    \let\Oldtex\TeX
    \let\Oldlatex\LaTeX
    \renewcommand{\TeX}{\textrm{\Oldtex}}
    \renewcommand{\LaTeX}{\textrm{\Oldlatex}}
    % Document parameters
    % Document title
    \title{EFR\_ES4\_Angoli\_Solari}
    
    
    
    
    
% Pygments definitions
\makeatletter
\def\PY@reset{\let\PY@it=\relax \let\PY@bf=\relax%
    \let\PY@ul=\relax \let\PY@tc=\relax%
    \let\PY@bc=\relax \let\PY@ff=\relax}
\def\PY@tok#1{\csname PY@tok@#1\endcsname}
\def\PY@toks#1+{\ifx\relax#1\empty\else%
    \PY@tok{#1}\expandafter\PY@toks\fi}
\def\PY@do#1{\PY@bc{\PY@tc{\PY@ul{%
    \PY@it{\PY@bf{\PY@ff{#1}}}}}}}
\def\PY#1#2{\PY@reset\PY@toks#1+\relax+\PY@do{#2}}

\@namedef{PY@tok@w}{\def\PY@tc##1{\textcolor[rgb]{0.73,0.73,0.73}{##1}}}
\@namedef{PY@tok@c}{\let\PY@it=\textit\def\PY@tc##1{\textcolor[rgb]{0.25,0.50,0.50}{##1}}}
\@namedef{PY@tok@cp}{\def\PY@tc##1{\textcolor[rgb]{0.74,0.48,0.00}{##1}}}
\@namedef{PY@tok@k}{\let\PY@bf=\textbf\def\PY@tc##1{\textcolor[rgb]{0.00,0.50,0.00}{##1}}}
\@namedef{PY@tok@kp}{\def\PY@tc##1{\textcolor[rgb]{0.00,0.50,0.00}{##1}}}
\@namedef{PY@tok@kt}{\def\PY@tc##1{\textcolor[rgb]{0.69,0.00,0.25}{##1}}}
\@namedef{PY@tok@o}{\def\PY@tc##1{\textcolor[rgb]{0.40,0.40,0.40}{##1}}}
\@namedef{PY@tok@ow}{\let\PY@bf=\textbf\def\PY@tc##1{\textcolor[rgb]{0.67,0.13,1.00}{##1}}}
\@namedef{PY@tok@nb}{\def\PY@tc##1{\textcolor[rgb]{0.00,0.50,0.00}{##1}}}
\@namedef{PY@tok@nf}{\def\PY@tc##1{\textcolor[rgb]{0.00,0.00,1.00}{##1}}}
\@namedef{PY@tok@nc}{\let\PY@bf=\textbf\def\PY@tc##1{\textcolor[rgb]{0.00,0.00,1.00}{##1}}}
\@namedef{PY@tok@nn}{\let\PY@bf=\textbf\def\PY@tc##1{\textcolor[rgb]{0.00,0.00,1.00}{##1}}}
\@namedef{PY@tok@ne}{\let\PY@bf=\textbf\def\PY@tc##1{\textcolor[rgb]{0.82,0.25,0.23}{##1}}}
\@namedef{PY@tok@nv}{\def\PY@tc##1{\textcolor[rgb]{0.10,0.09,0.49}{##1}}}
\@namedef{PY@tok@no}{\def\PY@tc##1{\textcolor[rgb]{0.53,0.00,0.00}{##1}}}
\@namedef{PY@tok@nl}{\def\PY@tc##1{\textcolor[rgb]{0.63,0.63,0.00}{##1}}}
\@namedef{PY@tok@ni}{\let\PY@bf=\textbf\def\PY@tc##1{\textcolor[rgb]{0.60,0.60,0.60}{##1}}}
\@namedef{PY@tok@na}{\def\PY@tc##1{\textcolor[rgb]{0.49,0.56,0.16}{##1}}}
\@namedef{PY@tok@nt}{\let\PY@bf=\textbf\def\PY@tc##1{\textcolor[rgb]{0.00,0.50,0.00}{##1}}}
\@namedef{PY@tok@nd}{\def\PY@tc##1{\textcolor[rgb]{0.67,0.13,1.00}{##1}}}
\@namedef{PY@tok@s}{\def\PY@tc##1{\textcolor[rgb]{0.73,0.13,0.13}{##1}}}
\@namedef{PY@tok@sd}{\let\PY@it=\textit\def\PY@tc##1{\textcolor[rgb]{0.73,0.13,0.13}{##1}}}
\@namedef{PY@tok@si}{\let\PY@bf=\textbf\def\PY@tc##1{\textcolor[rgb]{0.73,0.40,0.53}{##1}}}
\@namedef{PY@tok@se}{\let\PY@bf=\textbf\def\PY@tc##1{\textcolor[rgb]{0.73,0.40,0.13}{##1}}}
\@namedef{PY@tok@sr}{\def\PY@tc##1{\textcolor[rgb]{0.73,0.40,0.53}{##1}}}
\@namedef{PY@tok@ss}{\def\PY@tc##1{\textcolor[rgb]{0.10,0.09,0.49}{##1}}}
\@namedef{PY@tok@sx}{\def\PY@tc##1{\textcolor[rgb]{0.00,0.50,0.00}{##1}}}
\@namedef{PY@tok@m}{\def\PY@tc##1{\textcolor[rgb]{0.40,0.40,0.40}{##1}}}
\@namedef{PY@tok@gh}{\let\PY@bf=\textbf\def\PY@tc##1{\textcolor[rgb]{0.00,0.00,0.50}{##1}}}
\@namedef{PY@tok@gu}{\let\PY@bf=\textbf\def\PY@tc##1{\textcolor[rgb]{0.50,0.00,0.50}{##1}}}
\@namedef{PY@tok@gd}{\def\PY@tc##1{\textcolor[rgb]{0.63,0.00,0.00}{##1}}}
\@namedef{PY@tok@gi}{\def\PY@tc##1{\textcolor[rgb]{0.00,0.63,0.00}{##1}}}
\@namedef{PY@tok@gr}{\def\PY@tc##1{\textcolor[rgb]{1.00,0.00,0.00}{##1}}}
\@namedef{PY@tok@ge}{\let\PY@it=\textit}
\@namedef{PY@tok@gs}{\let\PY@bf=\textbf}
\@namedef{PY@tok@gp}{\let\PY@bf=\textbf\def\PY@tc##1{\textcolor[rgb]{0.00,0.00,0.50}{##1}}}
\@namedef{PY@tok@go}{\def\PY@tc##1{\textcolor[rgb]{0.53,0.53,0.53}{##1}}}
\@namedef{PY@tok@gt}{\def\PY@tc##1{\textcolor[rgb]{0.00,0.27,0.87}{##1}}}
\@namedef{PY@tok@err}{\def\PY@bc##1{{\setlength{\fboxsep}{-\fboxrule}\fcolorbox[rgb]{1.00,0.00,0.00}{1,1,1}{\strut ##1}}}}
\@namedef{PY@tok@kc}{\let\PY@bf=\textbf\def\PY@tc##1{\textcolor[rgb]{0.00,0.50,0.00}{##1}}}
\@namedef{PY@tok@kd}{\let\PY@bf=\textbf\def\PY@tc##1{\textcolor[rgb]{0.00,0.50,0.00}{##1}}}
\@namedef{PY@tok@kn}{\let\PY@bf=\textbf\def\PY@tc##1{\textcolor[rgb]{0.00,0.50,0.00}{##1}}}
\@namedef{PY@tok@kr}{\let\PY@bf=\textbf\def\PY@tc##1{\textcolor[rgb]{0.00,0.50,0.00}{##1}}}
\@namedef{PY@tok@bp}{\def\PY@tc##1{\textcolor[rgb]{0.00,0.50,0.00}{##1}}}
\@namedef{PY@tok@fm}{\def\PY@tc##1{\textcolor[rgb]{0.00,0.00,1.00}{##1}}}
\@namedef{PY@tok@vc}{\def\PY@tc##1{\textcolor[rgb]{0.10,0.09,0.49}{##1}}}
\@namedef{PY@tok@vg}{\def\PY@tc##1{\textcolor[rgb]{0.10,0.09,0.49}{##1}}}
\@namedef{PY@tok@vi}{\def\PY@tc##1{\textcolor[rgb]{0.10,0.09,0.49}{##1}}}
\@namedef{PY@tok@vm}{\def\PY@tc##1{\textcolor[rgb]{0.10,0.09,0.49}{##1}}}
\@namedef{PY@tok@sa}{\def\PY@tc##1{\textcolor[rgb]{0.73,0.13,0.13}{##1}}}
\@namedef{PY@tok@sb}{\def\PY@tc##1{\textcolor[rgb]{0.73,0.13,0.13}{##1}}}
\@namedef{PY@tok@sc}{\def\PY@tc##1{\textcolor[rgb]{0.73,0.13,0.13}{##1}}}
\@namedef{PY@tok@dl}{\def\PY@tc##1{\textcolor[rgb]{0.73,0.13,0.13}{##1}}}
\@namedef{PY@tok@s2}{\def\PY@tc##1{\textcolor[rgb]{0.73,0.13,0.13}{##1}}}
\@namedef{PY@tok@sh}{\def\PY@tc##1{\textcolor[rgb]{0.73,0.13,0.13}{##1}}}
\@namedef{PY@tok@s1}{\def\PY@tc##1{\textcolor[rgb]{0.73,0.13,0.13}{##1}}}
\@namedef{PY@tok@mb}{\def\PY@tc##1{\textcolor[rgb]{0.40,0.40,0.40}{##1}}}
\@namedef{PY@tok@mf}{\def\PY@tc##1{\textcolor[rgb]{0.40,0.40,0.40}{##1}}}
\@namedef{PY@tok@mh}{\def\PY@tc##1{\textcolor[rgb]{0.40,0.40,0.40}{##1}}}
\@namedef{PY@tok@mi}{\def\PY@tc##1{\textcolor[rgb]{0.40,0.40,0.40}{##1}}}
\@namedef{PY@tok@il}{\def\PY@tc##1{\textcolor[rgb]{0.40,0.40,0.40}{##1}}}
\@namedef{PY@tok@mo}{\def\PY@tc##1{\textcolor[rgb]{0.40,0.40,0.40}{##1}}}
\@namedef{PY@tok@ch}{\let\PY@it=\textit\def\PY@tc##1{\textcolor[rgb]{0.25,0.50,0.50}{##1}}}
\@namedef{PY@tok@cm}{\let\PY@it=\textit\def\PY@tc##1{\textcolor[rgb]{0.25,0.50,0.50}{##1}}}
\@namedef{PY@tok@cpf}{\let\PY@it=\textit\def\PY@tc##1{\textcolor[rgb]{0.25,0.50,0.50}{##1}}}
\@namedef{PY@tok@c1}{\let\PY@it=\textit\def\PY@tc##1{\textcolor[rgb]{0.25,0.50,0.50}{##1}}}
\@namedef{PY@tok@cs}{\let\PY@it=\textit\def\PY@tc##1{\textcolor[rgb]{0.25,0.50,0.50}{##1}}}

\def\PYZbs{\char`\\}
\def\PYZus{\char`\_}
\def\PYZob{\char`\{}
\def\PYZcb{\char`\}}
\def\PYZca{\char`\^}
\def\PYZam{\char`\&}
\def\PYZlt{\char`\<}
\def\PYZgt{\char`\>}
\def\PYZsh{\char`\#}
\def\PYZpc{\char`\%}
\def\PYZdl{\char`\$}
\def\PYZhy{\char`\-}
\def\PYZsq{\char`\'}
\def\PYZdq{\char`\"}
\def\PYZti{\char`\~}
% for compatibility with earlier versions
\def\PYZat{@}
\def\PYZlb{[}
\def\PYZrb{]}
\makeatother


    % For linebreaks inside Verbatim environment from package fancyvrb. 
    \makeatletter
        \newbox\Wrappedcontinuationbox 
        \newbox\Wrappedvisiblespacebox 
        \newcommand*\Wrappedvisiblespace {\textcolor{red}{\textvisiblespace}} 
        \newcommand*\Wrappedcontinuationsymbol {\textcolor{red}{\llap{\tiny$\m@th\hookrightarrow$}}} 
        \newcommand*\Wrappedcontinuationindent {3ex } 
        \newcommand*\Wrappedafterbreak {\kern\Wrappedcontinuationindent\copy\Wrappedcontinuationbox} 
        % Take advantage of the already applied Pygments mark-up to insert 
        % potential linebreaks for TeX processing. 
        %        {, <, #, %, $, ' and ": go to next line. 
        %        _, }, ^, &, >, - and ~: stay at end of broken line. 
        % Use of \textquotesingle for straight quote. 
        \newcommand*\Wrappedbreaksatspecials {% 
            \def\PYGZus{\discretionary{\char`\_}{\Wrappedafterbreak}{\char`\_}}% 
            \def\PYGZob{\discretionary{}{\Wrappedafterbreak\char`\{}{\char`\{}}% 
            \def\PYGZcb{\discretionary{\char`\}}{\Wrappedafterbreak}{\char`\}}}% 
            \def\PYGZca{\discretionary{\char`\^}{\Wrappedafterbreak}{\char`\^}}% 
            \def\PYGZam{\discretionary{\char`\&}{\Wrappedafterbreak}{\char`\&}}% 
            \def\PYGZlt{\discretionary{}{\Wrappedafterbreak\char`\<}{\char`\<}}% 
            \def\PYGZgt{\discretionary{\char`\>}{\Wrappedafterbreak}{\char`\>}}% 
            \def\PYGZsh{\discretionary{}{\Wrappedafterbreak\char`\#}{\char`\#}}% 
            \def\PYGZpc{\discretionary{}{\Wrappedafterbreak\char`\%}{\char`\%}}% 
            \def\PYGZdl{\discretionary{}{\Wrappedafterbreak\char`\$}{\char`\$}}% 
            \def\PYGZhy{\discretionary{\char`\-}{\Wrappedafterbreak}{\char`\-}}% 
            \def\PYGZsq{\discretionary{}{\Wrappedafterbreak\textquotesingle}{\textquotesingle}}% 
            \def\PYGZdq{\discretionary{}{\Wrappedafterbreak\char`\"}{\char`\"}}% 
            \def\PYGZti{\discretionary{\char`\~}{\Wrappedafterbreak}{\char`\~}}% 
        } 
        % Some characters . , ; ? ! / are not pygmentized. 
        % This macro makes them "active" and they will insert potential linebreaks 
        \newcommand*\Wrappedbreaksatpunct {% 
            \lccode`\~`\.\lowercase{\def~}{\discretionary{\hbox{\char`\.}}{\Wrappedafterbreak}{\hbox{\char`\.}}}% 
            \lccode`\~`\,\lowercase{\def~}{\discretionary{\hbox{\char`\,}}{\Wrappedafterbreak}{\hbox{\char`\,}}}% 
            \lccode`\~`\;\lowercase{\def~}{\discretionary{\hbox{\char`\;}}{\Wrappedafterbreak}{\hbox{\char`\;}}}% 
            \lccode`\~`\:\lowercase{\def~}{\discretionary{\hbox{\char`\:}}{\Wrappedafterbreak}{\hbox{\char`\:}}}% 
            \lccode`\~`\?\lowercase{\def~}{\discretionary{\hbox{\char`\?}}{\Wrappedafterbreak}{\hbox{\char`\?}}}% 
            \lccode`\~`\!\lowercase{\def~}{\discretionary{\hbox{\char`\!}}{\Wrappedafterbreak}{\hbox{\char`\!}}}% 
            \lccode`\~`\/\lowercase{\def~}{\discretionary{\hbox{\char`\/}}{\Wrappedafterbreak}{\hbox{\char`\/}}}% 
            \catcode`\.\active
            \catcode`\,\active 
            \catcode`\;\active
            \catcode`\:\active
            \catcode`\?\active
            \catcode`\!\active
            \catcode`\/\active 
            \lccode`\~`\~ 	
        }
    \makeatother

    \let\OriginalVerbatim=\Verbatim
    \makeatletter
    \renewcommand{\Verbatim}[1][1]{%
        %\parskip\z@skip
        \sbox\Wrappedcontinuationbox {\Wrappedcontinuationsymbol}%
        \sbox\Wrappedvisiblespacebox {\FV@SetupFont\Wrappedvisiblespace}%
        \def\FancyVerbFormatLine ##1{\hsize\linewidth
            \vtop{\raggedright\hyphenpenalty\z@\exhyphenpenalty\z@
                \doublehyphendemerits\z@\finalhyphendemerits\z@
                \strut ##1\strut}%
        }%
        % If the linebreak is at a space, the latter will be displayed as visible
        % space at end of first line, and a continuation symbol starts next line.
        % Stretch/shrink are however usually zero for typewriter font.
        \def\FV@Space {%
            \nobreak\hskip\z@ plus\fontdimen3\font minus\fontdimen4\font
            \discretionary{\copy\Wrappedvisiblespacebox}{\Wrappedafterbreak}
            {\kern\fontdimen2\font}%
        }%
        
        % Allow breaks at special characters using \PYG... macros.
        \Wrappedbreaksatspecials
        % Breaks at punctuation characters . , ; ? ! and / need catcode=\active 	
        \OriginalVerbatim[#1,codes*=\Wrappedbreaksatpunct]%
    }
    \makeatother

    % Exact colors from NB
    \definecolor{incolor}{HTML}{303F9F}
    \definecolor{outcolor}{HTML}{D84315}
    \definecolor{cellborder}{HTML}{CFCFCF}
    \definecolor{cellbackground}{HTML}{F7F7F7}
    
    % prompt
    \makeatletter
    \newcommand{\boxspacing}{\kern\kvtcb@left@rule\kern\kvtcb@boxsep}
    \makeatother
    \newcommand{\prompt}[4]{
        {\ttfamily\llap{{\color{#2}[#3]:\hspace{3pt}#4}}\vspace{-\baselineskip}}
    }
    

    
    % Prevent overflowing lines due to hard-to-break entities
    \sloppy 
    % Setup hyperref package
    \hypersetup{
      breaklinks=true,  % so long urls are correctly broken across lines
      colorlinks=true,
      urlcolor=urlcolor,
      linkcolor=linkcolor,
      citecolor=citecolor,
      }
    % Slightly bigger margins than the latex defaults
    
    \geometry{verbose,tmargin=1in,bmargin=1in,lmargin=1in,rmargin=1in}
    
    

\begin{document}
    
    \maketitle
    
    

     
            
    
    \begin{center}
    \adjustimage{max size={0.9\linewidth}{0.9\paperheight}}{EFR_ES4_Angoli_Solari_files/EFR_ES4_Angoli_Solari_1_0.png}
    \end{center}
    { \hspace*{\fill} \\}
    

    \hypertarget{energetica-e-fonti-rinnovabili---denerg-politecnico-di-torino}{%
\section*{Energetica e Fonti Rinnovabili - DENERG Politecnico di
Torino}\label{energetica-e-fonti-rinnovabili---denerg-politecnico-di-torino}}
\addcontentsline{toc}{section}{Energetica e Fonti Rinnovabili - DENERG
Politecnico di Torino}

    \begin{center}\rule{0.5\linewidth}{0.5pt}\end{center}

    \hypertarget{esercitazione-4---angoli-solari}{%
\section{Esercitazione 4 - Angoli
solari}\label{esercitazione-4---angoli-solari}}

    \hypertarget{obiettivi-di-apprendimento}{%
\subsection{Obiettivi di
apprendimento}\label{obiettivi-di-apprendimento}}

\begin{itemize}
\tightlist
\item
  Ricavare l'ora solare in uno specifico luogo del pianeta avendo a
  disposizione: l'ora del luogo, le sue coordinate (latitudine e
  longitudine, e longitudine del meridiano di riferimento) e il giorno
  dell'anno.
\item
  Leggere una carta solare (o diagramma solare) per un dato luogo (per
  una certa latitudine) e a ricavare graficamente gli angoli solari
  azimut e altitudine solare essendo noti: il giorno dell'anno e l'ora
  solare. Ricavare graficamente le ore di alba e tramonto.
\item
  Ricavare analiticamente gli angoli solari azimut e altitudine solare
  essendo noti: le coordinate del luogo (latitudine), il giorno
  dell'anno e l'ora solare. Confrontare i risultati analitici con quelli
  ottenuti dal diagramma solare.
\item
  Calcolare analiticamente l'ora di alba e tramonto in un determinato
  luogo del pianeta in un giorno specifico avendo a disposizione: le
  coordinate del luogo e la data. Confrontare i risultati analitici con
  quelli ottenuti dal diagramma solare.
\end{itemize}

    \hypertarget{esercizi}{%
\subsection{Esercizi}\label{esercizi}}

\begin{enumerate}
\def\labelenumi{\arabic{enumi}.}
\tightlist
\item
  Un osservatore si trova a Torino (Latitudine \(L\) 45.00 N(+),
  longitudine \(l\) 7.68 E(-)) il giorno \textbf{30 Marzo 2022}. Sapendo
  che il meridiano di riferimento per Torino è quello dell'Etna
  (longitudine standard \(l_{st}\) 15.00 E(-)) e che l'orologio segna le
  ore \textbf{13:00}, calcolare l'ora solare corrispondente.
  \textbf{Nota bene: il 30 Marzo è attiva l'ora legale!}
\item
  Viene fornito il diagramma solare di Torino (valido per la latitudine
  45.00 N). Leggere tramite il diagramma quanto valgono gli angoli
  solari azimut e altitudine solare, ipotizzando che il giorno dell'anno
  sia lo stesso dell'esercizio (1) e che l'ora solare sia la medesima
  ricavata. Determinare anche l'ora di alba e tramonto per lo stesso
  giorno utilizzando il diagramma solare fornito. A partire dalle ore
  solari ottenute, ricavare le ore locali corrispondenti.
\end{enumerate}
 
            
    
    \begin{center}
    \adjustimage{max size={0.9\linewidth}{0.9\paperheight}}{EFR_ES4_Angoli_Solari_files/EFR_ES4_Angoli_Solari_7_0.png}
    \end{center}
    { \hspace*{\fill} \\}
    

    \begin{enumerate}
\def\labelenumi{\arabic{enumi}.}
\setcounter{enumi}{2}
\tightlist
\item
  Ricavare analiticamente gli angoli di azimut e altitudine solare
  visualizzati da un osservatore che si trova a Torino lo stesso giorno
  dell'esercizio (1) quando l'ora solare è quella ricavata nel medesimo.
  Confrontare i risultati ottenuti con quelli letti graficamente
  nell'esercizio (2). Nota: calcolare analiticamente tutti gli angoli
  solari fondamentali non forniti nel testo. Nota: utilizzando il
  diagramma solare fornito nell'esercizio (2), scegliere quale formula
  analitica utilizzare per il calcolo dell'azimut.
\item
  Calcolare analiticamente l'ora solare di alba e tramonto per lo stesso
  giorno dell'esercizio (1) a Torino. Confrontare gli orari ottenuti con
  quelli che si possono leggere dal diagramma solare ed ottenuti
  nell'esercizio (2).
\end{enumerate}

    \newpage

    
    \hypertarget{esercizio-1}{%
\subsection{Esercizio 1}\label{esercizio-1}}

Un osservatore si trova a Torino (Latitudine \(L\) 45.00 N(+),
longitudine \(l\) 7.68 E(-)) il giorno \textbf{30 Marzo 2022}. Sapendo
che il meridiano di riferimento per Torino è quello dell'Etna
(longitudine standard \(l_{st}\) 15.00 E(-)) e che l'orologio segna le
ore \textbf{13:00}, calcolare l'ora solare corrispondente. \textbf{Nota
bene: il 30 Marzo è attiva l'ora legale!}

    \hypertarget{dati}{%
\subsubsection{Dati:}\label{dati}}
 
            
    
    \(L = 45^{\circ};\ l_{local} = -7.68^{\circ};\ l_{st} = -15^{\circ};\ ora = 13.0 \ h;\  n = 89\)

    

    \hypertarget{procedimento}{%
\subsubsection{Procedimento}\label{procedimento}}

L'ora solare in ogni luogo si riferisce al tempo esatto in cui il sole
attraversa il piano nord-sud equivalente al meridiano centrale del posto
(mezzogiorno solare) dove raggiunge anche la culminazione (la massima
elevazione, o altitudine). Il tempo solare differisce dall'ora di un
comune orologio a causa del fuso orario del meridiano di riferimento e
dall'equazione del tempo.

Per calcolare l'ora solare, o \emph{Solar Time} \(ST\), si applica
l'equazione seguente:
\[ST = LST + ET + (l_{st}-l_{local})\cdot4 \quad [min] \quad (1)\]
\(LST\) è il \textbf{Local Standard Time}, cioè l'ora locale riferita al
fuso orario in cui si trova l'osservatore (tempo civile o l'ora di
orologio). Il fuso orario è una porzione longitudinale della superficie
terrestre compresa tra due determinati meridiani che adotta lo stesso
orario per scopi legali, economici e sociali
(\href{https://it.wikipedia.org/wiki/Fuso_orario\#/media/File:World_Time_Zones_Map.png}{qui}
mappa dei fusi orari). A tale scopo si è divisa la superficie del globo
in 24 fusi orari, ognuno limitato da due meridiani geografici distanti
tra loro 15°, e si è stabilito che in ognuno di questi fusi tutti i
paesi adottino il tempo solare medio corrispondente al meridiano
centrale (tempo civile). La convenzione è stata adottata dalla
maggioranza degli Stati, con la modificazione che le linee di divisione
tra un fuso e l'altro seguono, anziché le linee longitudinali,
l'andamento dei confini delle singole nazioni. Il fuso orario è a sua
volta riferito al tempo coordinato universale \emph{UTC}, cioè il fuso
orario scelto come riferimento globale (longitudine equivalente nella
zona del Meridiano fondamentale di Greenwich), a partire dal quale sono
calcolati tutti i fusi orari del mondo.

    In Italia si adotta il fuso orario \emph{Central European Time CET}
corrispondente a \emph{UTC+1} e riferito al meridiano dell'Etna (-15°).
Inoltre, durante il periodo estivo vige l'ora legale, dunque sarà
necessario aggiungere un'ora al \emph{CET} (si chiamerà \emph{Central
European Summer Time CEST}). In figura, la posizione dell'osservatore
(in base alla latitudine e longitudine forniti) rispetto la terra:
 
            
    
    \begin{center}
    \adjustimage{max size={0.9\linewidth}{0.9\paperheight}}{EFR_ES4_Angoli_Solari_files/EFR_ES4_Angoli_Solari_15_0.png}
    \end{center}
    { \hspace*{\fill} \\}
    

    Nel caso giorno in esame, è attiva l'ora legale, per cui:
 
            
    
    \(LST = ora - 1 = 12.0 \ h = 720.0 \ min\).

    

    L'\textbf{equazione del tempo} \(ET\) è lo scostamento tra il tempo
indicato da un orologio solare rispetto al suo scorrere regolare
indicato da un orologio. Tale scostamento è la conseguenza dell'azione
combinata tra l'inclinazione dell'asse rispetto all'eclittica
(declinazione solare) e l'eccentricità dell'orbita della Terra attorno
al Sole (\href{https://it.wikipedia.org/wiki/Equazione_del_tempo}{wiki}
per approfondire):
 
            
    
    \begin{center}
    \adjustimage{max size={0.9\linewidth}{0.9\paperheight}}{EFR_ES4_Angoli_Solari_files/EFR_ES4_Angoli_Solari_19_0.png}
    \end{center}
    { \hspace*{\fill} \\}
    

    \begin{quote}
\textbf{Nota}: \href{https://www.geogebra.org/m/nwhgeqkz}{Qui} un
grafico interattivo del modello semplificato sole-terra.
\end{quote}

    L'equazione del tempo \(ET\) si calcola con la seguente equazione
approssimata in funzione del numero del giorno \(n\):

    \[\begin{align*}
ET &= 9.87\sin 2B -7.53\cos B -1.5\sin B  \quad [min] \quad (2)\\
\text{dove:}\\
B &= \frac{360(n-81)}{364} \quad [^{{\circ}}] \quad (3)\\
\end{align*}\]

    \begin{quote}
\textbf{Attenzione}: B è espresso in gradi, quindi gli argomenti delle
funzioni seno e coseno sono in gradi!
\end{quote}

    Di seguito il grafico di \(ET\) in funzione del giorno \(n\):

    \begin{center}
    \adjustimage{max size={0.9\linewidth}{0.9\paperheight}}{EFR_ES4_Angoli_Solari_files/EFR_ES4_Angoli_Solari_25_0.png}
    \end{center}
    { \hspace*{\fill} \\}
    
    Nel giorno in esame, usando le equazioni (2) e (3) otteniamo:
 
            
    
    \(B = 7.912^{\circ}=0.13809 \ rad \implies ET = -4.97 \ min\).

    

    L'ultimo argomento dell'equazione (1) è la \textbf{correzione
longitudinale}, necessaria per tenere conto della diversa longitudine
dell'osservatore rispetto al meridiano standard, e dove il fattore \(4\)
sono \(min/^{\circ}\), corrispondente a \(60 \ min\) ogni \(15^{\circ}\)
(il tempo che impiega la terra a ruotare rispetto la distanza di due
meridiani consecutivi). Applicando la correzione otteniamo:
 
            
    
    \((l_{st}-l_{local})\cdot4 = -29.28 \ min\).

    

    Infine, calcolati tutti i termini dell'equazione (1) e facendo
attenzione a riportarli tutti in minuti, possiamo calcolare l'ora solare
\(ST\):
 
            
    
    \(ST = 685.75 \min\) =\textgreater{} \(11.43 \ h\) =\textgreater{}
\(11{:}26\).

    
 
            
    
    Dai risultati, si può notare come l'ora locale \(LST\) differisce
dall'ora solare \(ST\) di ben \(00{:}34=34.0 \min\).

    

    \begin{quote}
\textbf{Nota}: Una rappresentazione visiva dell'effetto dell'equazione
del tempo e della declinazione terrestre è l'analemma, cioè una
particolare curva geometrica a forma di otto che descrive la posizione
del Sole nei diversi giorni dell'anno, alla stessa ora e nella località
o meridiano dell'osservatore
(\href{https://www.geogebra.org/m/vd3qpad8}{qui} un esempio di Analemma
interattivo).
\end{quote}

    \hypertarget{esercizio-2}{%
\subsection{Esercizio 2}\label{esercizio-2}}

Viene fornito il diagramma solare di Torino (valido per la latitudine
45.00 N). Leggere tramite il diagramma quanto valgono gli angoli solari
azimut e altitudine solare, ipotizzando che il giorno dell'anno sia lo
stesso dell'esercizio (1) e che l'ora solare sia la medesima ricavata.
Determinare anche l'ora di alba e tramonto per lo stesso giorno
utilizzando il diagramma solare fornito. A partire dalle ore solari
ottenute, ricavare le ore locali corrispondenti.

    \hypertarget{procedimento}{%
\subsubsection{Procedimento}\label{procedimento}}

    L'esercizio viene risolto utilizzando i risultati dell'esercizio
precedente ed il diagramma solare fornito durante l'esercitazione
(valido per Latitudine 45° N):
 
            
    
    \begin{center}
    \adjustimage{max size={0.9\linewidth}{0.9\paperheight}}{EFR_ES4_Angoli_Solari_files/EFR_ES4_Angoli_Solari_37_0.png}
    \end{center}
    { \hspace*{\fill} \\}
    

    Dall'analisi del diagramma otteniamo i seguenti risultati:

\begin{itemize}
\tightlist
\item
  Declinazione \(\delta_s \simeq 3.462^{{\circ}}\) (per interpolazione
  lineare tra Mar.~21 0° e Apr.~3 +5°)
\item
  Azimuth \(a_s \simeq -12^{{\circ}}\)
\item
  Altitudine solare \(\alpha \simeq 48^{{\circ}}\)
\item
  Alba \(ST_{sr} \simeq\) 05:40 =\textgreater{} \(LST_{sr} \simeq\)
  06:14
\item
  Tramonto \(ST_{ss} \simeq\) = 18:10 =\textgreater{}
  \(LST_{ss} \simeq\) 18:44
\end{itemize}

    \hypertarget{esercizio-3}{%
\subsection{Esercizio 3}\label{esercizio-3}}

Ricavare analiticamente gli angoli di azimut e altitudine solare
visualizzati da un osservatore che si trova a Torino lo stesso giorno
dell'esercizio (1) quando l'ora solare è quella ricavata nel medesimo.
Confrontare i risultati ottenuti con quelli letti graficamente
nell'esercizio (2). Nota: calcolare analiticamente tutti gli angoli
solari fondamentali non forniti nel testo. Nota: utilizzando il
diagramma solare fornito nell'esercizio (2), scegliere quale formula
analitica utilizzare per il calcolo dell'azimut.

    \hypertarget{procedimento}{%
\subsubsection{Procedimento}\label{procedimento}}

    Per calcolare analiticamente gli angoli solari \emph{azimuth} \(a_s\) e
\emph{altitudine solare} \(\alpha\) (o il complementare \emph{zenith}
\(\theta\)) è necessario conoscere le relazioni trigonometriche che
legano i suddetti angoli solari (riferimento sole-osservatore) con gli
angoli solari fondamentali (riferimento sole-terra): \emph{latitudine}
\(L\), \emph{declinazione} \(\delta_s\) e \emph{angolo orario} \(h_s\).
Di seguito un'illustrazione degli angoli solari:
 
            
    
    \begin{center}
    \adjustimage{max size={0.9\linewidth}{0.9\paperheight}}{EFR_ES4_Angoli_Solari_files/EFR_ES4_Angoli_Solari_42_0.jpg}
    \end{center}
    { \hspace*{\fill} \\}
    

    La \textbf{latitudine} \(L\) definisce la posizione dell'osservatore
rispetto all'equatore.

La \textbf{declinazione solare} \(\delta_s\) è l'angolo tra la linea dei
raggi solari, cioè il piano eclittico, e il piano passante per
l'equatore. Come visto a lezione, è un angolo compreso tra -23.45° (sud
equatore) e +23.45° (nord equatore). Infatti, la declinazione è una
conseguenza dell'inclinazione dell'asse terrestre e della sua orbita
attorno al Sole. Di seguito una rappresentazione della declinazione
solare:
 
            
    
    \begin{center}
    \adjustimage{max size={0.9\linewidth}{0.9\paperheight}}{EFR_ES4_Angoli_Solari_files/EFR_ES4_Angoli_Solari_44_0.png}
    \end{center}
    { \hspace*{\fill} \\}
    

    \begin{quote}
\textbf{Nota}: \href{https://www.geogebra.org/m/BJxyY5Vz}{Qui} un
grafico interattivo della radiazione solare verso la terra in base alla
declinazione e alla posizione dell'osservatore.
\end{quote}

    La declinazione \(\delta_s\) può essere calcolata in funzione del giorno
dell'anno \(n\), secondo l'equazione (formula di Cooper):
\[\delta_s = 23.45^{{\circ}}\sin\left( \frac{360^{{\circ}}(284+n)}{365} \right) \quad [^{{\circ}}]\]

    \begin{quote}
\textbf{Attenzione}: l'argomento del seno è espresso in gradi!
\(360^{{\circ}}= 2\pi \ rad\).
\end{quote}

    Di seguito il grafico di \(\delta_s\) in funzione di \(n\):

    \begin{center}
    \adjustimage{max size={0.9\linewidth}{0.9\paperheight}}{EFR_ES4_Angoli_Solari_files/EFR_ES4_Angoli_Solari_49_0.png}
    \end{center}
    { \hspace*{\fill} \\}
     
            
    
    Al giorno \(n=89\) la declinazione solare risulta
\(\delta_s=3.219^{\circ}\).

    

    L'\textbf{angolo orario} \(h_s\) esprime in gradi la distanza oraria
(espressa in ore decimali o minuti) tra l'ora solare \(ST\) ed il
mezzogiorno solare vero 12:00 (cioè l'ora solare in cui il sole si trova
nel piano nord-sud, o meridiano, e raggiunge la culminazione):
\[h_s = 15[^{{\circ}}/h]\cdot(ST[h]-12)=\frac{ST[min]-12\cdot60}{4[min/^{{\circ}}]} \quad [^{{\circ}}] \quad (3)\]
 
            
    
    L'ora solare calcolata nell'esercizio (1) è \(ST=11.43 \ h\),
equivalente ad un angolo orario solare \(h_s = -8.56^{\circ}\).

    

    L'\textbf{altitudine solare} \(\alpha\) è l'angolo verticale tra una
linea collineare con i raggi del sole e il piano orizzontale
dell'osservatore, e si calcola attraverso la seguente relazione
trigonometrica:
\[\sin \alpha=\sin L \sin \delta_s + \cos L \cos \delta_s \cos h_s \quad (4)\]

    \begin{quote}
\textbf{Nota}: l'angolo complementare a 90° dell'altitudine solare
\(\alpha\) è l'\textbf{angolo solare di zenith} \(\theta\) (o
semplicemente \(z\)), cioè l'angolo tra linea del sole e la linea
verticale all'osservatore (chiamato appunto zenith), corrispondente alla
perpendicolare al piano orizzontale. Dunque \(z=90^{{\circ}}-\alpha\), e
per le proprietà trigonometriche vale
\(\cos \alpha = \cos(90^{{\circ}}-z) = \sin z\).
\end{quote}
 
            
    
    Risolvendo si ottiene \(\alpha = 0.8298\ rad=47.55^{\circ}\).

    

    L'\textbf{azimuth solare} \(a_s\) è l'angolo tra la linea Sud e la
proiezione sul piano orizzontale della linea che idealmente congiunge
l'osservatore e il Sole, e si calcola attraverso la seguente relazione
trigonometrica:
\[\sin a_s  = \frac{\cos \delta_s \sin h_s}{\cos \alpha } \quad (5)\]

    \begin{quote}
\textbf{Attenzione}: valida solo se \(|a_s|<90^{{\circ}}\).
\end{quote}

    Per capire se \(|a_s|>90^{{\circ}}\) possiamo utilizzare il diagramma
solare e visivamente individuare le zone del traggitto del sole in cui
si troverà verso Nord, oltre la linea est-ovest. Possiamo notare dal
diagramma solare che il sole si trova verso Sud e dunque ci troviamo
sicuramente nella condizione \(|a_s|<90^{{\circ}}\).

In alternativa, se \(|L|>|\delta_s|\) possiamo calcolare analiticamente
il valore assoluto dell'angolo orario \(h_{s,ew}\) in cui il sole
intercetta il piano est-ovest, cioè nel momento in cui l'azimuth solare
\(a_s\) è pari a +90° o -90° (il sole si troverà nella linea est/ovest).
Dunque, imponendo \(a_s=90^{{\circ}}\) e risolvendo il sistema di
equazioni (4) e (5) per l'angolo orario \(h_{s,ew}\) otteniamo:
\[a_s = 90^{{\circ}} \implies h_{s,ew} = \cos^{-1}\left(\cot L \tan \delta_s \right) \quad [rad]\]
Se risulta \(|h_s|<h_{s,ew}\) allora siamo certi che
\(|a_s|<90^{{\circ}}\).

Se invece ci troviamo ai tropici e \(|L|\leq|\delta_s|\), il sole
rimarrà sempre a Nord (Sud nell'emisfero Sud) rispetto la linea
est/ovest ed il valore di \(a_s\) sarà sempre maggiore (minore) di 90°,
dunque l'espressione per il calcolo di \(h_{s,ew}\) non è più valida.

    Quando \(|a_s|>90^{{\circ}}\) si dovrà utilizzare la formula corretta
vista a lezione (riportata nella sezione Extra per comodità).
 
            
    
    Nel nostro caso
\(h_{s,ew}=86.78^{\circ}>|h_s| \implies |a_s|<90^{\circ}\).

    
 
            
    
    Dunque, utilizzando la formula (5) e risolvendo otteniamo
\(a_s=-0.2221\ rad= -12.72^{\circ}\).

    

    \hypertarget{esercizio-4}{%
\subsection{Esercizio 4}\label{esercizio-4}}

Calcolare analiticamente l'ora solare di alba e tramonto per lo stesso
giorno dell'esercizio (1) a Torino. Confrontare gli orari ottenuti con
quelli che si possono leggere dal diagramma solare ed ottenuti
nell'esercizio (2).

\hypertarget{procedimento}{%
\subsubsection{Procedimento}\label{procedimento}}

Per il calcolo dell'ora solare di alba e tramonto basterà utilizzare la
formula (4) trovando l'angolo orario ed imporre \(\alpha=0^\circ\), cioè
quando il sole si trova nel piano orizzontale dell'osservatore:
\[\alpha=0° \implies h_{ss} \ \text{or} \ h_{sr} = \pm \cos^{-1}\left ( -\tan L \tan \delta_s \right ) \quad [rad]\]
 
            
    
    Otteniamo \(h_{ss}=93.22^{\circ}\) e \(h_{sr}=-93.22^{\circ}\).

    

    \begin{quote}
\textbf{Nota}: La relazione per il calcolo degli angoli orari di alba e
tramonto considera il centro del cerchio solare sull'orizzonte. In altre
parole, alba e tramonto sono definiti come i momenti in cui l'arco
superiore del sole si trova nell'orizzonte.
\end{quote}
 
            
    
    Per il calcolo delle ore solari corrispondenti basterà applicare la
formula (3). Dunque avremo \(ST_{ss}=18.21\ h\) e \(ST_{sr}=5.79\ h\).

    
 
            
    
    Convertendo in HH:MM otteniamo \(ST_{ss}=18{:}13\) e
\(ST_{sr}=05{:}47\).

    
 
            
    
    Infine, utilizzando l'equazione (1) possiamo ottenere l'ora locale di
alba e tramondo: \(LST_{ss}=18{:}47\) e \(LST_{sr}=06{:}21\).

    

    \hypertarget{confronto-metodo-grafico-esercizio-2-con-metodo-analitico-esercizi-34}{%
\subsection{Confronto metodo grafico Esercizio 2 con metodo analitico
Esercizi
3/4}\label{confronto-metodo-grafico-esercizio-2-con-metodo-analitico-esercizi-34}}

\begin{longtable}[]{@{}lcc@{}}
\toprule
& Metodo analitico & Metodo Grafico \\
\midrule
\endhead
\(\delta_s\) & 3.22° & 3.46° \\
\(\alpha\) & 47.55° & 48° \\
\(a_s\) & -12.72° & -12° \\
\(ST_{ss}\) & 18:13 & 18:10 \\
\(ST_{sr}\) & 05:47 & 05:40 \\
\bottomrule
\end{longtable}

    \hypertarget{extra}{%
\subsection{Extra}\label{extra}}

\hypertarget{carta-solare-con-python}{%
\subsubsection{Carta solare con Python}\label{carta-solare-con-python}}

    E' possibile ricostruire il diagramma solare utilizzando librerie
specifiche in Python, e.g.~\emph{pvlib}. Di seguito un elaborazione del
diagramma per la località e la data scelta in input. Il diagramma solare
potrebbe differire leggermente da quello visto a lezione a causa di un
diverso sistema di riferimento. In questo caso si nota che: i) il raggio
rappresenta lo zenith piuttosto che l'altezza solare; ii) il diagramma è
riferito all'orario locale \(LST\) (civile, escludendo l'ora legale) e
per tale ragione sono presenti gli analemmi, necessari per tenere conto
dell'ora solare e dunque della reale posizione del sole. Fissata l'ora
\(LST\), la colorbar ci aiuta ad identificare in quale punto
dell'analemma ci troviamo in base al giorno dell'anno, quindi
identificando esattamente la declinazione, lo zenith e l'azimuth solare.

    \begin{center}
    \adjustimage{max size={0.9\linewidth}{0.9\paperheight}}{EFR_ES4_Angoli_Solari_files/EFR_ES4_Angoli_Solari_71_0.png}
    \end{center}
    { \hspace*{\fill} \\}
    
    Nel seguente diagramma si evidenzia la posizione del sole nel giorno e
nell'ora considerata, con l'indicazione dell'ora di alba e tramonto.

    \begin{center}
    \adjustimage{max size={0.9\linewidth}{0.9\paperheight}}{EFR_ES4_Angoli_Solari_files/EFR_ES4_Angoli_Solari_73_0.png}
    \end{center}
    { \hspace*{\fill} \\}
    
    Confronto risultati libreria Python con risultati ottenuti in classe:

    \begin{Verbatim}[commandchars=\\\{\}]
pvlib:
apparent\_elevation     48.165810
elevation              48.151471
azimuth               166.780096
apparent\_zenith        41.834190
zenith                 41.848529
solar\_time             11.413582
Name: 2022-03-30 12:00:00, dtype: float64

Ora solare calcolata: 11.43 h
Equazione del tempo libreria: -4.98 min
Equazione del tempo calcolata: -4.97 min
Declinazione eq. cooper (usata in classe): 3.22 °
Declinazione eq. spencer: 3.47 °
    \end{Verbatim}

    \begin{quote}
\textbf{Nota}: la libreria \emph{pvlib} utilizza delle formulazioni più
precise per il calcolo della posizione del sole (azimuth, zenith e
declinazione) e dell'ora solare rispetto alle formule utilizzate in
classe (l'azimuth solare è calcolato con riferimo 0° a Nord e valori
crescenti positivi in senso orario). Nonostante ciò, si può notare come
le formule utilizzate rappresentino comunque un'ottima approssimazione.
I valori apparenti di altitudine e zenith solari sono riferiti al bordo
superiore del sole quando tocca l'orizzonte.
\end{quote}

    \hypertarget{tool-online}{%
\subsubsection{Tool online}\label{tool-online}}

    Esistono diversi strumenti online che permettono di calcolare la
posizione del sole ed il suo percorso durante il giorno.
 
            
    
    Tra questi, su
\href{https://www.suncalc.org/\#/45,7.68,18/2022.03.30/13:00/0/1}{suncalc}
potrete visualizzare il percorso del sole (e tante altre utili
informazioni) direttamente su mappa satellitare con i dati di input al
notebook.

    

    \hypertarget{correzione-formula-calcolo-azimuth-solare-a_s}{%
\subsubsection{\texorpdfstring{Correzione formula calcolo azimuth solare
\(a_s\)}{Correzione formula calcolo azimuth solare a\_s}}\label{correzione-formula-calcolo-azimuth-solare-a_s}}

Per risolvere il problema di validità della formula (5) (valida solo per
\(|a_s|<90^{{\circ}}\)) sarà necessario correggerla. La seguente
formulazione generalizzata è valida per qualsiasi angolo di azimuth
solare in qualsisi quadrante e per qualsiasi giorno ed orario:

    \[\begin{align*}
a_s &= \sigma_{ew} \sigma_{ns} a_{s_0} +\left (  \frac{1-\sigma_{ew}\sigma_{ns}}{2}\right ) \sigma_w\cdot 180^{{\circ}}\\
\text{dove:} \\
a_{s0} &=\sin^{-1}\left ( \frac{\cos \delta_s \sin h_s }{\cos \alpha} \right ) \quad [^{{\circ}}] \\
h_{s,ew}  &=  \begin{cases} \cos^{-1}\left(\cot L \tan \delta_s \right), & \text{se} \ |L|>|\delta_s| \\ M, & \text{altrimenti} \  \end{cases}\\
\sigma_{ew} &= \begin{cases} 1, & \text{se} \ |h_s|< h_{s,ew} \\ -1, & \text{altrimenti} \end{cases} \\
% \sigma_{ns}  &= \begin{cases} 1, & \text{se} \ L(L-\delta_s)\geq0 \\ -1, & \text{altrimenti} \end{cases} \\
\sigma_{ns}  &= \begin{cases} 1, & \text{se} \ L>\delta_s \\ -1, & \text{altrimenti} \end{cases} \\
\sigma_w &= \begin{cases} 1, & \text{se} \ h_s\geq 0 \\ -1, & \text{altrimenti} \end{cases} \\
\end{align*}\]

    \begin{quote}
\textbf{Nota}: \(M\) è un numero molto grande tale che si verifichi
sempre \(|h_s|< h_{s,ew}\).
\end{quote}

    Esempio di applicazione riferito all'esercizio (1):
 
            
    
    \(h_s = -8.56^{\circ}; \ h_{s,ew} = 86.78^{\circ}; \ a_{s0} = -12.72^{\circ}; \ \sigma_{ew} = 1; \ \sigma_{ns} = 1; \ \sigma_w = -1; \ a_s = -12.72^{\circ}\)

    

    \begin{center}\rule{0.5\linewidth}{0.5pt}\end{center}

Le versioni statiche ed aggiornate dei notebook le trovate online su
\href{https://github.com/DSSchiera/materials}{\includegraphics{https://badgen.net/badge/icon/github?icon=github\&label}}.

Le versioni interattive dei notebook le trovate su
\href{https://mybinder.org/v2/gh/DSSchiera/materials/HEAD}{\includegraphics{https://mybinder.org/badge_logo.svg}}.


    % Add a bibliography block to the postdoc
    
    
    
\end{document}
